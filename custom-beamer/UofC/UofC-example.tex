\documentclass{beamer}

\title{Interdependence and Difference}
\subtitle{Niche Construction and the Study of Medicine}
\author{Timothy Elder}
\institute
{
  Department of Sociology \\
  University of Chicago
}

\usetheme{UofC}
\usefonttheme{UofC} % titling and headers
\usefonttheme{serif} % This sets the main font to Minion Pro

\begin{document}

\begin{frame}
\titlepage
\end{frame}
\begin{frame}{The Workflow}
    \begin{figure}
    \includegraphics{/Users/timothyelder/Documents/plaintext_workshop/figures/workflow-wide-v2.png}
    \end{figure}
\end{frame}

\begin{frame}{The Workflow}{Explained}
    The workflow looks complicated but it is actually straight forward. This is what we do: \newline
    \begin{enumerate}
        \pause
        \item Write our documents in Markdown
        \pause
        \item Tell \texttt{pandoc} to typeset our document
        \pause
        \item \texttt{pandoc} uses LaTeX to create our document
    \end{enumerate}
\end{frame}

\begin{frame}{The Workflow}{Simplified}
    \begin{figure}
    \includegraphics[width = \textwidth]{/Users/timothyelder/Documents/plaintext_workshop/figures/md2tex2pdf.png}
    \end{figure}
\end{frame}

\begin{frame}
    \Huge{MARKUP LANGUAGES} \newline
    \large{\LaTeX}
\end{frame}

\begin{frame}{Markup Languages}{\LaTeX}

    Symbolically represents:
    \begin{itemize}
        \item Style
        \item Formatting
        \item Relationships of Document Sections\newline
    \end{itemize}

    \footnotesize{e.g. HTML, XML, Markdown, \LaTeX}

\end{frame}

\begin{frame}[fragile]{Markup Languages}{\LaTeX}

    \begin{columns}
    
    \begin{column}{0.48\textwidth}
    \color{red}\rule{\linewidth}{4pt}
    Styled
    \begin{itemize}
        \item \textit{Italicized}
        \item \textbf{Bolded}
        \item \underline{Underlined}
    \end{itemize}
    \end{column}
    
    \begin{column}{0.48\textwidth}
    \color{blue}\rule{\linewidth}{4pt}
    Plain Text
    \begin{itemize}
        \item \texttt{\textbackslash textit\{Italicized\}}
        \item \texttt{\textbackslash textbf\{Bolded\}} 
        \item \texttt{\textbackslash underline\{Underlined\}}
    \end{itemize}
    \end{column}
    
    \end{columns}

\end{frame}

\begin{frame}[fragile]{Markup Languages}{\LaTeX}
    \begin{scriptsize}
    \begin{lstlisting}
        \begin{columns}
            \begin{column}{0.48\textwidth}
            \color{red}\rule{\linewidth}{4pt}
            Styled
            \begin{itemize}
                \item \textit{Italicized}
                \item \textbf{Bolded}
                \item \underline{Underlined}
            \end{itemize}
            \end{column}
            
            \begin{column}{0.48\textwidth}
            \color{blue}\rule{\linewidth}{4pt}
            Plain Text
            \begin{itemize}
                \item \texttt{\textbackslash textit\{Italicized\}}
                \item \texttt{\textbackslash textbf\{Bolded\}} 
                \item \texttt{\textbackslash underline\{Underlined\}}
            \end{itemize}
            \end{column}
        \end{columns}
\end{lstlisting}
\end{scriptsize}
\end{frame}

\begin{frame}{Markup Languages}{\LaTeX}
    \begin{figure}
    \includegraphics[width = \textwidth]{/Users/timothyelder/Documents/plaintext_workshop/figures/tex_eg.png}
    \end{figure}
\end{frame}

\begin{frame}{Markup Languages}{\LaTeX}
    Every LaTeX Document has Three Parts:
    \begin{enumerate}
        \item Macros and Libraries
        \item Preamble
        \item Body
    \end{enumerate}
\end{frame}

\begin{frame}{Markup Languages}{\LaTeX}
    Every LaTeX Document has Three Parts:
    \begin{enumerate}
        \item Macros and Libraries (Not Visible)
        \item Preamble
        \item Body
    \end{enumerate}
\end{frame}

\begin{frame}
    \Huge{MARKUP LANGUAGES} \newline
    \large{Markdown}
\end{frame}

\begin{frame}{Markup Languages}{Markdown}
    
        \begin{description}[Other description]
        \item[Source File]
        A file which contains the data and recipe for creating some output. A
        source file includes your Markdown Draft, LaTeX file, or python/R
        scripts with data analysis. Think of it as the input that will produce
        the output.
        \end{description}
    
\end{frame}

\begin{frame}{Markup Languages}{Markdown}
    \begin{small}
        \begin{table}[]
            \begin{tabular}{ll}
            \multicolumn{1}{c}{\textbf{Element}} & \multicolumn{1}{c}{\textbf{Markdown Syntax}}                                           \\ \hline
            \multicolumn{1}{l|}{\Large{Heading}}         & \begin{tabular}[c]{@{}l@{}}\texttt{\# H1}\\ \texttt{\#\# H2}\\ \texttt{\#\#\# H3}\end{tabular}                    \\ \hline
            \multicolumn{1}{l|}{\textbf{Bold}}   & \texttt{**bold text**}                                                                          \\ \hline
            \multicolumn{1}{l|}{\textit{Italic}} & \texttt{*italicized text*}                                                                      \\ \hline
            \multicolumn{1}{l|}{Blockquote}      & \textgreater \texttt{blockquote}                                                                \\ \hline
            \multicolumn{1}{l|}{Ordered List}    & \begin{tabular}[c]{@{}l@{}}\texttt{1. First item}\\ \texttt{2. Second item}\\ \texttt{3. Third item}\end{tabular} \\ \hline
            \multicolumn{1}{l|}{Unordered List}  & \begin{tabular}[c]{@{}l@{}}\texttt{- First item}\\ \texttt{- Second item}\\ \texttt{- Third item}\end{tabular}    \\ \hline
            \multicolumn{1}{l|}{\texttt{Code}}   & \texttt{`code`}                                                                                 \\ \hline
            \multicolumn{1}{l|}{Horizontal Rule} & \--\--\--                                                                                    \\ \hline
            \multicolumn{1}{l|}{Link}            & \texttt{{[}title{]}(https://www.example.com)}                                                   \\ \hline
            \multicolumn{1}{l|}{Image}           & \texttt{!{[}alt text{]}(image.jpg)}                                                             \\ \hline
            \end{tabular}
            \end{table}
    \end{small}
\end{frame}

\begin{frame}
    \Huge{THANKS} \newline
    \large{For you to do:}
    \begin{enumerate}
        \item Sign up for the next Session, March 3rd
        \item Sign up for office hours, Tuesdays
    \end{enumerate}
\end{frame}


\end{document}