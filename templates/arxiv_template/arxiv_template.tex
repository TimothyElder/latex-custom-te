\documentclass{article}
\usepackage{arxiv}
\usepackage[utf8]{inputenc} % allow utf-8 input
\usepackage[T1]{fontenc}    % use 8-bit T1 fonts
\usepackage{hyperref}       % hyperlinks
\usepackage{url}            % simple URL typesetting
\usepackage{booktabs}       % professional-quality tables
\usepackage{amsfonts}       % blackboard math symbols
\usepackage{nicefrac}       % compact symbols for 1/2, etc.
\usepackage{microtype}      % microtypography
\usepackage{lipsum}		    % Can be removed after putting your text content
\usepackage{graphicx}
\usepackage{natbib}
\usepackage{doi}

\title{A template for the \emph{arxiv} style}

%\date{September 9, 1985}	% Here you can change the date presented in the paper title
%\date{} 					% Or removing it

\author{ \href{https://orcid.org/0000-0000-0000-0000}{\includegraphics[scale=0.06]{orcid.pdf}\hspace{1mm}David S.~Hippocampus}\thanks{Use footnote for providing further
		information about author (webpage, alternative
		address)---\emph{not} for acknowledging funding agencies.} \\
	Department of Computer Science\\
	Cranberry-Lemon University\\
	Pittsburgh, PA 15213 \\
	\texttt{hippo@cs.cranberry-lemon.edu} \\
	%% examples of more authors
	\And
	\href{https://orcid.org/0000-0000-0000-0000}{\includegraphics[scale=0.06]{orcid.pdf}\hspace{1mm}Elias D.~Striatum} \\
	Department of Electrical Engineering\\
	Mount-Sheikh University\\
	Santa Narimana, Levand \\
	\texttt{stariate@ee.mount-sheikh.edu} \\
	%% \AND
	%% Coauthor \\
	%% Affiliation \\
	%% Address \\
	%% \texttt{email} \\
	%% \And
	%% Coauthor \\
	%% Affiliation \\
	%% Address \\
	%% \texttt{email} \\
	%% \And
	%% Coauthor \\
	%% Affiliation \\
	%% Address \\
	%% \texttt{email} \\
}

% Uncomment to remove the date
%\date{}

% Uncomment to override  the `A preprint' in the header
%\renewcommand{\headeright}{Technical Report}
%\renewcommand{\undertitle}{Technical Report}
\renewcommand{\shorttitle}{\textit{arXiv} Template}

%%% Add PDF metadata to help others organize their library
%%% Once the PDF is generated, you can check the metadata with
%%% $ pdfinfo template.pdf
\hypersetup{
pdftitle={A template for the arxiv style},
pdfsubject={q-bio.NC, q-bio.QM},
pdfauthor={David S.~Hippocampus, Elias D.~Striatum},
pdfkeywords={First keyword, Second keyword, More},
}

\begin{document}
\maketitle

\begin{abstract}
	\lipsum[1]
\end{abstract}


% keywords can be removed
\keywords{First keyword \and Second keyword \and More}


Status and Subfield:

The Distribution of Sociological Specializations across Departments

Timothy B. Elder

Austin C. Kozlowski

University of Chicago

ABSTRACT:

This study takes the well-established finding that sociology departments
are ordered by a stable status hierarchy and investigates the
implications of this hierarchy for the discipline's subfields. Using
data drawn from the 2020 American Sociological Association's \emph{Guide
to Graduate Departments}, we show that subfields are not uniformly
distributed across departments, but that certain subfields are
over-represented in elite departments while others are concentrated in
lower prestige institutions. We show that this ``status ordering of
subfields'' is tightly linked to key characteristics of the subfields
and their researchers. Most notably, subfields that are male-dominated
and theoretically-oriented are more highly represented at elite
departments than those that are more feminized or practically-oriented.
We also provide provisional evidence that subfields with greater African
American membership are concentrated at lower status institutions while
subfields with a higher proportion Asian membership tend to be
consolidated at elite departments. We conclude with a discussion of
potential implications for the research and researchers disadvantaged by
this implicit ordering.

\textbf{\uline{Introduction:}}

Sociology has a history of diagnosing inequities embedded in
institutions and analyzing how such disparities and the systems of power
within which they emerge shape the production of knowledge. These
disciplinary strengths position sociology to be a uniquely ``reflexive''
science, potentially capable of interpreting the conditions of its own
knowledge production with greater clarity than other sciences (Bourdieu
1988). In some respects, sociology does appear to be spared some of the
clearest scientific inequalities apparent in other disciplines;
sociology has near gender parity among faculty (American Sociological
Association 2020; Scelza, Spalter-Roth, and Mayorova 2011) and awards a
higher percentage doctorates to African American and Hispanic/Latino
students than economics or political science (National Science
Foundation 2020). Yet, this characterization of the discipline as a
whole masks its internal differentiation. Sociology is a fragmented
field, with relatively autonomous subfields pursuing distinct sets of
questions by a variety of methodologies. Hierarchies of status,
reputation, and authority are well documented \emph{between} scientific
disciplines, but similar hierarchies may also manifest \emph{within}
highly differentiated disciplines (Abbott 2001).

The purpose of this study is to investigate the status ordering of
sociology's subfields and to identify how this ordering relates to the
scholarly features of those subfields and the socio-demographic
characteristics of the researchers who pursue them. Drawing upon
complete faculty listings for more than 100 sociology departments as
published in the 2020 edition of the American Sociological Association's
(ASA) \emph{Guide to Graduate Departments}, we find that the
distribution of subfields across departments is intimately tied to the
durable departmental status ordering, with some subfields concentrated
in elite institutions and others in lower status departments.
Furthermore, we find that subfields that are more male and more
theoretical tend to be located in higher status departments than those
that are feminized or practically oriented. These findings illuminate a
status hierarchy in sociological knowledge production that typically
remains implicit, and aims to advance how we understand the social
conditions of our own knowledge production as well as the ways in which
certain researchers and fields of inquiry remain structurally
marginalized.

\textbf{\uline{Theoretical Framework:}}

\emph{Science and Status}

Status plays a central role in the institution of science, particularly
as it is carried out in universities (Merton 1979 {[}1942{]}). High
status departments aim to hire faculty with strong reputations for
scholarly excellence and promising prospects for future contributions,
with the hope that such successful hires will raise the department's
standing. Researchers, receiving no remuneration for publishing in
academic journals or amassing citations, are incentivized to produce
impactful research because of the gain in personal status. Indeed, in
fields where many researchers rarely acquire large external grants, the
only direct way a university benefits from its faculty's publications
lies in the gains to prestige associated with their research. Thus, in
the idealized model, prestige serves as a mechanism for meritocracy in
science.

However, a vast literature examining science and beyond, consistently
finds that status is not a pure reflection of achievement or scholarly
contribution, but is predictably concentrated within certain areas of
work and certain subpopulations (Gauchat and Kenneth 2018). For example,
work that is more manual, practical, and feminized is found to be lower
status than work that is perceived to be more cerebral, creative, and
masculine, with nursing and doctoring being key examples of such a
relationship (Allen 1997). This devaluing of feminized and practical
work has potential implications for the resources allocated to these
areas, the attention it garners, and the experiences of researchers in
these fields. The further possibility of scientific ``blind-spots'' due
to the devaluing of ``feminized'' areas of science can have further
consequences for the advancement of questions of social importance.

While previous research has documented prestige differences between
disciplines, there is reason to expect that similar patterns of status
differentiation exist \emph{within} disciplines. Abbott (2001) argues
that the social organization of scientific fields should be understood
as following fractal geometry, with self-similar patterns of opposition
repeating within subfields nested within fields. Bourdieu (1984)
similarly described social spaces with a nested geometry, with the
patterns of opposition structuring differences within institutions
mirroring those patterns of opposition that structure differences
between institutions. Moreover, his analysis of the French social
sciences in the 1960s suggested that status hierarchies manifest not
only between fields, like philosophy and sociology, but within the
fields themselves (Bourdieu 1988). Although many of the differentiations
associated with status hierarchies \emph{between} disciplines are
reproduced within disciplines, there has been little systematic
investigation of how these hierarchies manifest between the subfields of
a single discipline.

\emph{The Subfields of Sociology}

Sociology serves as a uniquely instructive case for the analysis of
intra-disciplinary status orders. Sociology is a highly differentiated
field, encompassing a broad and loosely connected array of topics,
methodologies, and theoretical orientations. Unlike economics which is
primarily ordered around a theoretical framework (rational-choice),
anthropology which is ordered around a methodological intervention
(ethnography) and political science which is ordered around a topical
area (politics), sociology is oriented to a loose set of areas of
interests and several competing theoretical frameworks. The internal
differentiation of the discipline into different areas of research
interests is further organized by the characteristics of the researchers
that comprise those areas.

The basic tenet of the sociology of science is that the formation of
knowledge is conditioned and channeled by the social conditions of
knowledge production. The kinds of claims that are made, the standards
of validity, and the evaluation of knowledge's value are all influenced
by the historical context of knowledge production and the social
position of those producing that knowledge (Bourdieu 1988; Collins 1998;
Shapin 1994; Latour 1999). To practice a truly reflexive sociology, it
is critical that sociologists consider the social organization of their
own knowledge production. Bourdieu posited that sociology is a uniquely
equipped science able to develop a reflexive practice which incorporates
in its knowledge production an understanding of its own social
situatedness. This study aims to advance reflexive social science by
providing a systematic description of the relations between subfields of
sociological knowledge and the social positions of the researchers that
compose them.

Lastly, a central concern of contemporary sociology is the
identification of systemic inequalities embedded within institutions.
Yet sociologists have dedicated relatively little attention to the
inequalities embedded in the discipline itself. Previous researchers
have shown a durable status hierarchy in sociology departments that has
largely persisted for the past century and which tightly structures
hiring between institutions (Burris 2004, Weakliem et a. 2011). However,
there has been no systematic investigation into whether this status
ordering of departments relates to sociology's differentiated areas of
research. By empirically rendering the relationship between this
well-documented status hierarchy of departments and areas of
sociological inquiry, this study aims to identify which domains of
sociological research are marginalized and which are privileged in the
American university system.

\textbf{\uline{Data:}}

We analyze data collected by the American Sociological Association (ASA)
between October and December 2019 and published in the 2020 edition of
the \emph{American Sociological Association's Guide to Graduate
Departments}. The \emph{Guide}, published each year by the ASA, provides
current overviews of almost 200 graduate departments of sociology,
including a complete roster of the current faculty and listing each
individual's PhD institution and subfields of study.

We limit our analysis to full-time faculty (excluding emeritus, adjunct,
and joint-appointed faculty) at PhD-granting departments at universities
in the United States, resulting in 2084 faculty across 111 universities.
The overwhelming majority of PhD granting departments submit their
faculty listings to the ASA \emph{Guide} each year; comparing the list
of PhD institutions of current faculty listed in the \emph{Guide}, we
find that 90.2\% of the faculty in our dataset obtained their PhDs from
one of the departments in the Guide\footnote{It is important to note
  that a small fraction of the Sociologists in our dataset received
  their PhD from non-US sociology departments, or from a
  non-sociological discipline. Our coverage of 90.2\% is calculated
  taking into account only the focal departments which constitute our
  analyses, these being PhD granting departments in Sociology located in
  the US.}. This makes the \emph{Guide} a powerful but underutilized
resource for analyzing the relationship between individual researchers,
departments, and subfields in American sociology.

We begin by constructing a status ordering of sociology departments
following the approach used by Burris (2004). Burris conceptualized the
status ordering of departments as instantiated in the network of
exchanges of PhD graduates between departments. For each faculty member,
the ASA guide lists their current institution as well as their PhD
institution. Following Burris, we use this data to construct a directed
graph of PhD exchange, then use eigenvector centrality as a proxy for
department status (Bonacich 1987). We correlate this statistic with the
department scores collected by US News and World Report's 2020 survey of
current sociology faculty and find that the two measures correlate at a
level of 0.84.

The subfields of research interest listed in the \emph{Guide} closely
(but imperfectly) mirror the set of formal ASA ``sections.'' When a
research interest listed for a faculty member does not explicitly
reference a section in the American Sociological Association we code the
research area into the most likely candidate section. The two authors of
this paper independently coded each interest and deliberated to achieve
consensus on those interests that produced divergent codes (less than
5\%).

Having measured ``departmental status scores,'' we proceed to identify
``subfield status scores,'' which capture the differential
representation of subfields across the departmental prestige hierarchy.
We produce this measure of subfield status by assigning each researcher
the prestige score of their current institution, then, for each
subfield, calculating the average prestige score of all researchers who
list this subfield among their interests. Thus, subfields that are
disproportionately represented among high status departments will earn a
higher ``subfield status score'' than those subfields predominantly
studied in lower status departments.

It is important to clarify that these subfield status scores do not
necessarily correspond to any socially shared subjective assessment of
subfields. Our measure of departmental status, by contrast, does appear
to correspond to subjective assessments, given its strong correlation
with the results from US News and World Report's survey of current
sociologists. However, we have no comparable subjective evaluation of
subfield prestige for comparison. Instead, we limit our analyses to the
systematic distribution of subfields across departments. Our claims
regarding ``subfield status'' are therefore not about subjective
assessments of subfield prestige, but the differential representation of
subfields across a departmental hierarchy that is measurable both
structurally (through PhD exchange networks) and inter-subjectively
(through surveying sociologists).

After identifying the status distribution of subfields, we assess the
relationship of this ordering to several social attributes that are
commonly intertwined with status. First, we assess the relationship
between a subfield's status position and its gender composition.
Feminist scholarship has long shown that objects and practices
associated with women are consistently devalued across a variety of
domains, including work (Reskin and Bielby, 2005), entertainment, art,
and modes of self-presentation (Goffmann, 1977). Studies of science also
consistently find that areas of inquiry with greater percentage women
tend to be discounted and devalued (Rawlings and Bourgeois 2004). To
assess the relationship between a subfield's gender composition and
status, we collect data on the percentage male in each subfield from the
ASA's website.\footnote{The gender categories listed in ASA data are
  ``male,'' ``female,'' ``transgender,'' ``non-binary,'' and ``other.''
  We adopt this exact terminology to remain consistent with the data.}
Our analyses specifically use gender composition of ``regular members''
in each section, which excludes students and is therefore more
consistent with the measures we draw from the ASA Guide.

Prior research suggests that work performed by gender minorities
including transgender, nonbinary, and gender non-conforming researchers
may similarly suffer lower status relative work associated with cis men.
Although ASA does measure gender minority identities in 2019, the size
of this subpopulation is too small for robust statistical inference.
Because prior theory and our own provisional evidence suggest that
gender minority status is generally associated with disadvantaged status
positions relative cis men, we operationalize gender composition as
``percent men,'' effectively pooling women and gender minorities.

We also assess whether the status ordering of subfields is associated
with the average age of its researchers. Like many institutions,
university departments privilege senior members over junior members.
Junior faculty lack tenure and occupy a more precarious position than
their senior colleagues. Senior faculty have also enjoyed more time to
build a professional reputation, accrue social capital, and shape their
department and the discipline in alignment with their own visions. For
these reasons, it is plausible that subfields populated by younger
scholars would be concentrated at lower prestige departments than those
occupied by older faculty. We calculate the ``average age'' for each
subfield by subtracting each faculty member's year of PhD graduation
from 2020 then calculating the mean within subfields.

In addition to researcher characteristics like gender and age, we also
consider scholarly attributes of the subfields that have been linked to
status distinctions. First, we consider the quantitative/qualitative
methodological divide that deeply structures sociological knowledge
production. Research on legitimacy in science consistently finds
quantification is a common technique to improve the legitimacy of a
claim, with quantification bestowing an air of impartiality,
objectivity, and scientism. To measure the association of subfields with
quantitative or qualitative methodologies, we count each subfield's
co-occurrence with ``quantitative methodology'' and ``qualitative
methodology'' among faculty's listed interests in the ASA Guide. We then
calculate a ``percent quantitative'' estimate by dividing the total
number of quantitative overlaps by the sum of quantitative and
qualitative overlaps for each subfield. Some of the smaller subfields
exhibit only one or two overlaps with quantitative or qualitative
methodology among their practitioners, producing fragile estimates. To
improve estimation, we treat Ethnomethodology \& Conversation Analysis
as qualitative methodology and Mathematical Sociology as quantitative
methodology, and add overlaps with these two subfields to our
qualitative and quantitative totals, respectively. We believe this
analytic decision improves estimation of the ``quantitativeness'' of the
subfields, but refraining from this pooling step and simply dropping
cases that have fewer than 3 overlaps produces substantively identical
findings.

Lastly, we consider whether the subfield is more theoretical or
practical in nature. Theories of status from Veblen to Bourdieu
emphasize that distance from toil and drudgery has historically been
central to demonstrations of privilege. Studies of science confirm that
areas of inquiry associated with technical, manual, and practical work
are commonly subordinated to more ``pure'' fields of research. In an
analysis of the differentiation of university agriculture programs in
the early 20th century, Rawlings and Bourgeois (2004) show that a
division of studies into research-oriented, ``disinterested'' fields and
technical fields oriented towards addressing community issues
corresponded with a movement of higher prestige departments to
legitimize these more theoretical domains as rightly ``scientific.'' To
capture such a practical/theoretical differentiation in sociology, we
count each subfield's co-occurrence with ``Theory'' and with
``Sociological Practice and Public Sociology'' among faculty interests,
and create a Theoretical Score by dividing the number of ``Theory''
overlaps by the sum of ``Theory'' and ``Sociological Practice''
overlaps.

\textbf{\uline{Results:}}

We begin our analysis by showing that research in the subfields of
sociology is patterned by the status of departments. In Figure 1, we
plot the average departmental status for researchers in each subfield,
along with the median and interquartile range. Because subfield status
is operationalized as the mean departmental status of researchers in
that subfield, we can directly compare these subfield status scores to
departmental status scores. We find that the score of roughly 0.4 among
the highest status subfields corresponds to well endowed, high profile
departments ranked in the top 10 by US News and World Report. By
contrast, the score of approximately 0.1 among the lowest status
subfields corresponds with large public R2 universities. This means that
the average researcher in one of the highest status subfields is likely
to be located in an elite R1 school, whereas the average researcher in a
lower status subfield is more likely to be located in a public R2
school.

\textbf{Figure 1.} Mean, median and interquartile range of departmental
status for each subfield, with number of faculty listing each subfield
as an interest in 2020.

\includegraphics[width=6.65885in,height=4.59196in]{media/image2.png}

\emph{Note:} Dots represent means, vertical lines represent medians.

Sociology's subfields do not exist independently but cluster together
into common groupings based on shared topics as well demographic
characteristics of the researchers who compose those areas. To capture
the patterning of subfield membership and its relation to the status
ordering displayed above, Figure 2 displays results from
multidimensional scaling (MDS) of the subfield-by-subfield co-membership
matrix created by the ASA. Subfields that share many members are
positioned close together while subfields that share few are located far
apart.

Meaningful areas of inquiry are differentiated into regions of the MDS
plot. On the right side, we see subfields concerned with politics and
economy alongside several historical and theoretical subfields. These
subfields are largely high status, with the exceptions of Peace and
Environmental. But more strikingly, these subfields are all
predominantly male.

Directly below the political/economic cluster, we see three subfields
that share a global economic focus (``Global'', ``Development,''
``Labor''), and continuing to the bottom center of the plot, we see a
concentration of subfields that focus on race and racial minorities.
Continuing clockwise, we find subfields that are related to health and
demography as well as sex, gender, and education. These sections on the
bottom and left side of the field, along with the race-related
subfields, are markedly more feminized than the political and economic
subfields on the right. We also see that, with a few exceptions, these
subfields exhibit either medium or low status scores.

\textbf{Figure 2.} Results from multidimensional scaling (MDS) of
subfield-by-subfield overlap matrix of shared membership in ASA
sections.

\includegraphics[width=5.56979in,height=5.45654in]{media/image1.png}

Lastly, we see at the top of the figure a set of small, high status
subfields that are all predominantly male. Taking the field as a whole,
we can see a general tendency of increasing status and increasing
proportion male as we move from the bottom left of the field to the top
right. This pattern provides further evidence that scholars are not
randomly or uniformly distributed between the discipline's subfields,
but that certain subfields are dominated by men and high prestige
departments while others are more feminized and more concentrated at
lower prestige institutions.

Next, we directly assess the apparent association between a subfield's
status and its gender composition. Results are presented in Figure 3. In
the left panel we see a clear, positive association between status and
percent male across subfields. However, as we described above, there are
numerous other characteristics of scholars and research areas that may
similarly correspond to this status ordering. In the right panel of
Figure 3 we plot the association between age and status and find that it
is similarly positive and strong, with subfields that have older
memberships being more tightly concentrated in high status departments.

\textbf{Figure 3.} Associations of subfields' status scores with gender
composition and age of members.

\includegraphics[width=6.23417in,height=4.81597in]{media/image6.png}

In Figure 4 we visualize the association between a subfield's status and
its racial composition. We examine each of the four largest racial
identities individually. In the first panel we map the relationship
between subfield status and percent membership white, and find it to be
negative but non-significant. The next three panels show the
associations between subfield status and percent African American,
percent Asian, and percent Hispanic/Latino(a), respectively. There is
one subfield for each racial group that appears as an outlier in its
respective analysis: Race \& Ethnicity is 25\% African American, Asia
and Asian Americans is 69\% Asian, and Latina/o sociology is 51\%
Hispanic/Latino(a). Because associations could be strongly influenced by
these single outliers, for each of these subpopulations we display the
line of best fit including and excluding the respective outlier.

\textbf{Figure 4.} Associations between racial composition variables and
status of subfields.

\includegraphics[width=6.19271in,height=6.19271in]{media/image3.png}

In the top right panel, we see a clear, negative association between
subfield status and percent African American membership. This
association is attenuated by the inclusion of the moderately high status
``Race \& Ethnicity'' subfield, but retains its negative direction. By
contrast, we see that percent Asian membership has a strongly positive
association with subfield status, but is slightly dampened by inclusion
of the Asia and Asian Americans subfield. Percent Hispanic/Latino(a)
shows no meaningful association with status regardless of the inclusion
of the outlying subfield.

Having analyzed the relationships between subfield status and the
socio-demographic attributes of the researchers in the subfield's
researchers, we now shift to analyzing how scholarly characteristics of
the subfield itself relate to status. Figure 5 displays the associations
of status with the theoretical/practical orientation of the subfield and
with the quantitative/qualitative tendency of the field. We see a strong
association between theoretical orientation and status, such that more
theoretically-inclined subfields (i.e. those with greater co-occurrence
with Theory) tend to be represented in higher status departments than
more practically-oriented subfields (i.e. those with greater
co-occurrences with Sociological Practice and Public Sociology). By
contrast, there is no meaningful association between the
qualitative/quantitative orientation of the subfield and its status.
Thus, while the ``scientistic'' nature of quantitative social science
may still accrue various benefits to practitioners, such as improved
access to large grants or recognition by a broader scientific community,
it is not associated with placement in higher prestige sociology
deparments. Rather, our evidence suggests that within quantitative and
qualitative domains, select subfields are highly represented by elite
departments while others are underrepresented.

\textbf{Figure 5.} Associations of subfield status with theoretical and
quantitative orientations.

\includegraphics[width=6.5in,height=3.90278in]{media/image4.png}

Having established several bivariate associations, we next use multiple
regression models to assess the independent effects of each attribute.
Due to the small number of cases available for analysis, we avoid
fitting models with the full battery of independent variables. Instead,
we introduce related variables together in sets, then fit a ``full
model'' containing only those measures found to have significant
independent effects in earlier models. Results are presented in Table 1.

\includegraphics[width=6.15724in,height=5.28542in]{media/image5.png}

The first model of Table 1 shows the adjusted effects of percent male
and mean age on subfield status. While both percent male and mean age
showed strong correlations with status in our analyses above, we see in
Model 1 that only percent male retains its significance when the two
effects are assessed simultaneously. Older memberships also tend to be
male dominated, and this finding suggests that it is only through its
association with gender composition that age exhibits an association
with status.

In Model 2 we examine the adjusted effects for racial identities, with
percent white as the excluded reference category. We find that none of
the racial composition variables in Model 2 display independent effects.
This is in part because Model 2 does not exclude the various outlier
cases discussed above for each racial identity group. Because the
associations between racial composition and subfield status require
dropping outliers inconsistent with the overall trend, we consider these
effects to be provisional and exclude them from our final model.

We next examine the effects of theoretical/practical and
quantitative/qualitative orientations on subfield status. We find that
more theoretical subfields display higher status scores independent of
quantitative/qualitative orientation. This is unsurprising, given that
quantitative/qualitative orientation has only a weak association with
status, and therefore little potential as a confounder. Model 4 displays
the adjusted effects of the two variables found in earlier models to
have independent effects on status: percent male and theoretical
orientation. When assessed together, both variables retain their
positive, significant associations with subfield status, suggesting that
neither factor's connection to status is reducible to the other.
Cumulatively, the results from all these models suggest that the
distribution of subfields across the departmental status hierarchy can
be largely predicted by their gender composition and
theoretical/practical orientation, with more male-dominated and more
theoretically-oriented subfields being disproportionately concentrated
in high status departments. Model 4's R\textsuperscript{2} value of
0.396 attests to the strength of these associations and the tightness of
the relationship between the departmental status ordering, gender, and
theoretical-orientation in the distribution of subfields.

\textbf{\uline{Conclusion:}}

This study takes the well-established finding that sociology departments
are ordered by a stable status hierarchy (Burris 2004) and investigates
the implications of this hierarchy for the discipline's subfields. We
show that subfields are not uniformly distributed across departments,
but that certain subfields are over-represented in elite departments
while others are over-represented in low prestige institutions. We show
that this ``status ordering of subfields'' is tightly linked to key
characteristics of the subfields and their researchers. Most notably,
subfields that are more male-dominated and theoretically-oriented are
more highly represented at elite departments than those that are more
feminized or practically-oriented.

Although we are not able to assess the effects of status position on
subfields, a close examination of our descriptive findings in light of
well-documented characteristics of academic hierarchy point toward
potential implications. First, subfields that are highly represented at
elite universities are likely to benefit from the superior resources
available to these institutions (Münch 2014). Elite universities have
greater internal resources to fund research agendas and greater capacity
to reduce researchers' teaching responsibilities. However, we note that
some of the fields that routinely secure the greatest funds from
external grants (Medical/Health, Aging) are relatively ``low status'' in
the ordering we present here. Moreover, the highest status subfields --
Ethnomethodology and Conversation Analysis, Rationality and Society,
Economic Sociology and Comparative Historical sociology -- rarely
acquire multi-million dollar grants to support large projects. The
``high status subfields'' therefore should not be interpreted as the
best resourced fields, but niches of theoretical elaboration that are
privileged by high status institutions, often in spite of meager
material prospects. Thus, concentration in high status departments can
serve as an alternate mode for financing areas of research that, due to
their low practical applicability, have little opportunity for external
funding. However, we note that certain theoretical fields -- especially
female-dominated ones like Sexualities, Bodies and Embodiment, and
Emotions -- are under-represented at elite departments and hence lack
both the external funding available to more practically-oriented fields
of inquiry as well as the internal funding associated with elite
institutions.

The etiology of the subfield status order is similarly outside the scope
of our empirical investigation, but patterns emerging from our findings
point to some plausible causes. Much of the patterning we observe is
explicable in economic terms. The most elite sociology departments
primarily train students for academic research positions, often at
comparably elite institutions. It is thus sustainable for these
institutions to teach courses and train students in niche areas of
theoretical inquiry that have a persistent home in the upper echelons of
the academy. However, less elite institutions train graduate students
for a variety of research and administrative positions outside of
academia, and much of this work lies in applied areas such as health and
criminology. Because they must prepare their students for careers
outside of academia, these institutions privilege more applied fields of
sociological work over the more purely theoretical.

However, the gender bias that we observe cannot be so easily explained
as a symptom of purely economic factors. Although female-dominated
subfields do tend to be more practical, our multiple regression models
suggest that gender composition retains an independent association with
status even after controlling for subfields' theoretical/practical
orientation. A vast literature from feminist science studies find that
feminized approaches to knowledge production are commonly undervalued
relative traditionally-masculine approaches (Anderson 1995; Keller
1984). Reinforcing the formal and economic subordination of women and
women's work are entrenched cultural biases against feminized objects
and practices. These cultural and structural forces are deeply
intertwined, mutually sustaining gender-based inequalities within
science and across social institutions.

This study's findings have important implications for understanding the
academic production of sociological knowledge. In an idealized model of
science, status is allocated to researchers and institutions in
proportion to their contributions to the collective advancement of
knowledge (Merton 1979 {[}1942{]}). Such an idealized model would not
anticipate that research pertaining to practical social issues or
research conducted primarily by women would be systematically relegated
to institutions of lower prestige. Not only does this patterning suggest
that researchers in practical and feminized fields lack the material and
social advantages associated with holding faculty positions in elite
departments, but that the fields of knowledge they pursue may inherit
these associated disadvantages. Moreover, the character of research in a
subfield may, in various subtle ways, reflect its position in the
academic structure. In his classic study of French academia, Bourideu
(1988) argues that the most radical and revolutionary voices of
mid-century French social thought occupied relatively peripheral
positions in the academic system, while the works of central actors were
comparatively conservative. Nuanced details of the content of subfield
research is beyond the scope of our study, but an overview of the status
order we identify confirms that many of sociology's most critical
subfields are disproportionately housed in lower prestige departments.
This suggests the intriguing possibility that status position not only
affects the resources and visibility afforded a subfield, but may steer
its intellectual priorities and research directions.

It is basic to the sociology of science that knowledge reflects the
social conditions of its production. We focus here on sociology not
because of its uniqueness, but to emphasize that such a system of
stratification can persist even within an institution astutely aware of
structural inequities and putatively committed to egalitarianism.
Moreover, this exercise in self-criticism aims to clarify how entire
fields of sociological knowledge are embedded differentially in a
hierarchical academic order, which we believe is a critical first-step
in understanding the relationship between the knowledge we produce as
sociologists and the social positions we occupy in the academic
structure. Creating a robust and truly reflexive sociology requires that
we systematically interrogate the conditions of our own knowledge
production, and illuminate the varied ways by which certain branches are
elevated and others systematically marginalized.

\textbf{\uline{References}}

Abbott, Andrew. 2001. \emph{Chaos of Disciplines.} Chicago, IL:
University of Chicago Press.

Allen, D. (1997) ``The nursing-medical boundary: a negotiated order?''.
\emph{Sociology of Health \& Illness}, Vol. 19: pp. 498-520.

Anderson, Elizabeth. 1995. ``Feminist epistemology: An interpretation
and a defense.'' \emph{Hypatia} 10(3): 50-84.

American Sociological Association. (2020). \emph{Guide to Graduate
Departments of Sociology.} Washington, DC: American Sociological
Association

Barbara F. Reskin and Denise D. Bielby. (2005) "A Sociological
Perspective on Gender and Career Outcomes" \emph{The Journal of Economic
Perspectives} Vol. 19 (No. 1): pp. 71-86

Bonacich, Phillip. (1987). ``Power and Centrality: A Family of
Measures.'' \emph{American Journal of Sociology} Vol. 92 (No. 5): pp.
1170-1182

Bourdieu, Pierre. 1984. \emph{Distinction: A Social Critique of the
Judgment of Taste}. Harvard, MA: Harvard University Press.

Bourdieu, Pierre. 1988. \emph{Homo Academicus}. Stanford, CA: Stanford
University Press.

Burris, Val. (2004) "The academic caste system: Prestige hierarchies in
PhD exchange networks." \emph{American Sociological Review} Vol 69 (No.
2): pp. 239-264.

Collins, Harry. (1998) ``The Meaning of Data: Open and Closed Evidential
Cultures in the Search for Gravitational Waves''. \emph{American Journal
of Sociology}. Vol. 104 (No 2); pp. 293--338

Gauchat, Gordon, and Kenneth T. Andrews. 2018. ``The cultural-cognitive
mapping of scientific professions.'' \emph{American Sociological Review}
83(3):567-595.

Goffmann, Erving. (1977). ``The Arrangement between the Sexes.''
\emph{Theory and Society} Vol. 4 (No. 3): pp. 301-331.

Hofstra, Bas, Vivek V. Kulkarni, Sebastian Munoz-Najar Galvez, Bryan He,
Dan Jurafsky, and Daniel A. McFarland. (2020). ``The
Diversity--Innovation Paradox in Science.'' \emph{Proceedings of the
National Academy of Sciences} 117(17):9284.

Keller, Evelyn Fox. 1984. \emph{A Feeling for the Organism: The Life and
Work of Barbara McClintock}. San

Francisco: W. H. Freeman.

Merton, Robert K. (1979 {[}1942{]}). ``The Normative Structure of
Science.'' in \emph{The Sociology of Science}, edited by Norman W.
Storer. Chicago, IL: University of Chicago Press\emph{.}

Münch, Richard. 2014. \emph{Academic capitalism: Universities in the
global struggle for excellence}. New York, New York: Routledge.

National Science Foundation. 2020. \emph{Survey of Earned Doctorates}.
Retrieved from
\href{https://ncses.nsf.gov/pubs/nsf22300/data-tables}{\uline{https://ncses.nsf.gov/pubs/nsf22300/data-tables}}.

Roos, Patricia A. (1977) "Occupational Feminization, Occupational
Decline? Sociology\textquotesingle s Changing Sex Composition."
\emph{The American Sociologist}, Vol. 28 (No. 1): pp. 75-88

Sauder, Michael, and Wendy Nelson Espeland. "The discipline of rankings:
Tight coupling and organizational change." \emph{American sociological
review} 74.1 (2009): 63-82.

Shapin, Steven. 1994. \emph{A Social History of Truth: Civility and
Science in 17th Century England}. Chicago, IL: University of Chicago
Press.

Scelza, Janene, Roberta Spalter-Roth, and Olga Mayorova. 2011. ``A
Decade of Change: ASA Membership from 2000 - 2010.'' Washington, DC:
American Sociological Association.

\uline{Appendix}

\textbf{Table A1}: Subfield names and abbreviations.

\begin{longtable}[]{@{}
  >{\raggedright\arraybackslash}p{(\columnwidth - 6\tabcolsep) * \real{0.3558}}
  >{\raggedright\arraybackslash}p{(\columnwidth - 6\tabcolsep) * \real{0.1554}}
  >{\raggedright\arraybackslash}p{(\columnwidth - 6\tabcolsep) * \real{0.3269}}
  >{\raggedright\arraybackslash}p{(\columnwidth - 6\tabcolsep) * \real{0.1619}}@{}}
\toprule()
\begin{minipage}[b]{\linewidth}\raggedright
Section
\end{minipage} & \begin{minipage}[b]{\linewidth}\raggedright
Abbreviation
\end{minipage} & \begin{minipage}[b]{\linewidth}\raggedright
Section
\end{minipage} & \begin{minipage}[b]{\linewidth}\raggedright
Abbreviation
\end{minipage} \\
\begin{minipage}[b]{\linewidth}\raggedright
Aging and the Life Course
\end{minipage} & \begin{minipage}[b]{\linewidth}\raggedright
Aging
\end{minipage} & \begin{minipage}[b]{\linewidth}\raggedright
International Migration
\end{minipage} & \begin{minipage}[b]{\linewidth}\raggedright
Immig
\end{minipage} \\
\begin{minipage}[b]{\linewidth}\raggedright
Altruism Morality and Social Solidarity
\end{minipage} & \begin{minipage}[b]{\linewidth}\raggedright
Altruism
\end{minipage} & \begin{minipage}[b]{\linewidth}\raggedright
Labor and Labor Movements
\end{minipage} & \begin{minipage}[b]{\linewidth}\raggedright
Labor
\end{minipage} \\
\begin{minipage}[b]{\linewidth}\raggedright
Animals and Society
\end{minipage} & \begin{minipage}[b]{\linewidth}\raggedright
Animals
\end{minipage} & \begin{minipage}[b]{\linewidth}\raggedright
Latina/o Sociology
\end{minipage} & \begin{minipage}[b]{\linewidth}\raggedright
Latin
\end{minipage} \\
\begin{minipage}[b]{\linewidth}\raggedright
Asia and Asian America
\end{minipage} & \begin{minipage}[b]{\linewidth}\raggedright
Asia
\end{minipage} & \begin{minipage}[b]{\linewidth}\raggedright
Law
\end{minipage} & \begin{minipage}[b]{\linewidth}\raggedright
Law
\end{minipage} \\
\begin{minipage}[b]{\linewidth}\raggedright
Body and Embodiment
\end{minipage} & \begin{minipage}[b]{\linewidth}\raggedright
Body
\end{minipage} & \begin{minipage}[b]{\linewidth}\raggedright
Marxist Sociology
\end{minipage} & \begin{minipage}[b]{\linewidth}\raggedright
Marx
\end{minipage} \\
\begin{minipage}[b]{\linewidth}\raggedright
Children and Youth
\end{minipage} & \begin{minipage}[b]{\linewidth}\raggedright
Youth
\end{minipage} & \begin{minipage}[b]{\linewidth}\raggedright
Mathematical Sociology
\end{minipage} & \begin{minipage}[b]{\linewidth}\raggedright
Math
\end{minipage} \\
\begin{minipage}[b]{\linewidth}\raggedright
Collective Behavior and Social Movements
\end{minipage} & \begin{minipage}[b]{\linewidth}\raggedright
SocMvmt
\end{minipage} & \begin{minipage}[b]{\linewidth}\raggedright
Medical Sociology
\end{minipage} & \begin{minipage}[b]{\linewidth}\raggedright
Medic
\end{minipage} \\
\begin{minipage}[b]{\linewidth}\raggedright
Communication Information Technology and Media Sociology
\end{minipage} & \begin{minipage}[b]{\linewidth}\raggedright
Comms
\end{minipage} & \begin{minipage}[b]{\linewidth}\raggedright
Mental Health
\end{minipage} & \begin{minipage}[b]{\linewidth}\raggedright
MentHlth
\end{minipage} \\
\begin{minipage}[b]{\linewidth}\raggedright
Community and Urban Sociology
\end{minipage} & \begin{minipage}[b]{\linewidth}\raggedright
Urban
\end{minipage} & \begin{minipage}[b]{\linewidth}\raggedright
Methodology
\end{minipage} & \begin{minipage}[b]{\linewidth}\raggedright
Method
\end{minipage} \\
\begin{minipage}[b]{\linewidth}\raggedright
Comparative Historical Sociology
\end{minipage} & \begin{minipage}[b]{\linewidth}\raggedright
CompHist
\end{minipage} & \begin{minipage}[b]{\linewidth}\raggedright
Organizations Occupations and Work
\end{minipage} & \begin{minipage}[b]{\linewidth}\raggedright
Orgs
\end{minipage} \\
\begin{minipage}[b]{\linewidth}\raggedright
Consumers and Consumption
\end{minipage} & \begin{minipage}[b]{\linewidth}\raggedright
Consump
\end{minipage} & \begin{minipage}[b]{\linewidth}\raggedright
Peace War and Social Conflict
\end{minipage} & \begin{minipage}[b]{\linewidth}\raggedright
Peace
\end{minipage} \\
\begin{minipage}[b]{\linewidth}\raggedright
Crime Law and Deviance
\end{minipage} & \begin{minipage}[b]{\linewidth}\raggedright
Crim
\end{minipage} & \begin{minipage}[b]{\linewidth}\raggedright
Political Economy of the World System
\end{minipage} & \begin{minipage}[b]{\linewidth}\raggedright
PolEcon
\end{minipage} \\
\begin{minipage}[b]{\linewidth}\raggedright
Culture
\end{minipage} & \begin{minipage}[b]{\linewidth}\raggedright
Culture
\end{minipage} & \begin{minipage}[b]{\linewidth}\raggedright
Political Sociology
\end{minipage} & \begin{minipage}[b]{\linewidth}\raggedright
Pol
\end{minipage} \\
\begin{minipage}[b]{\linewidth}\raggedright
Development
\end{minipage} & \begin{minipage}[b]{\linewidth}\raggedright
Dev
\end{minipage} & \begin{minipage}[b]{\linewidth}\raggedright
Population
\end{minipage} & \begin{minipage}[b]{\linewidth}\raggedright
Pop
\end{minipage} \\
\begin{minipage}[b]{\linewidth}\raggedright
Disability and Society
\end{minipage} & \begin{minipage}[b]{\linewidth}\raggedright
Disab
\end{minipage} & \begin{minipage}[b]{\linewidth}\raggedright
Qualitative Methodology
\end{minipage} & \begin{minipage}[b]{\linewidth}\raggedright
Qual
\end{minipage} \\
\begin{minipage}[b]{\linewidth}\raggedright
Drugs and Society
\end{minipage} & \begin{minipage}[b]{\linewidth}\raggedright
Drugs
\end{minipage} & \begin{minipage}[b]{\linewidth}\raggedright
Quantitative Methodology
\end{minipage} & \begin{minipage}[b]{\linewidth}\raggedright
Quant
\end{minipage} \\
\begin{minipage}[b]{\linewidth}\raggedright
Economic Sociology
\end{minipage} & \begin{minipage}[b]{\linewidth}\raggedright
Econ
\end{minipage} & \begin{minipage}[b]{\linewidth}\raggedright
Race Gender and Class
\end{minipage} & \begin{minipage}[b]{\linewidth}\raggedright
RaceClassGen
\end{minipage} \\
\begin{minipage}[b]{\linewidth}\raggedright
Education
\end{minipage} & \begin{minipage}[b]{\linewidth}\raggedright
Edu
\end{minipage} & \begin{minipage}[b]{\linewidth}\raggedright
Racial and Ethnic Minorities
\end{minipage} & \begin{minipage}[b]{\linewidth}\raggedright
Race
\end{minipage} \\
\begin{minipage}[b]{\linewidth}\raggedright
Emotions
\end{minipage} & \begin{minipage}[b]{\linewidth}\raggedright
Emot
\end{minipage} & \begin{minipage}[b]{\linewidth}\raggedright
Rationality and Society
\end{minipage} & \begin{minipage}[b]{\linewidth}\raggedright
Rational
\end{minipage} \\
\begin{minipage}[b]{\linewidth}\raggedright
Environmental Sociology
\end{minipage} & \begin{minipage}[b]{\linewidth}\raggedright
Envir
\end{minipage} & \begin{minipage}[b]{\linewidth}\raggedright
Religion
\end{minipage} & \begin{minipage}[b]{\linewidth}\raggedright
Relig
\end{minipage} \\
\begin{minipage}[b]{\linewidth}\raggedright
Ethnomethodology and Conversation Analysis
\end{minipage} & \begin{minipage}[b]{\linewidth}\raggedright
Ethno
\end{minipage} & \begin{minipage}[b]{\linewidth}\raggedright
Science Knowledge and Technology
\end{minipage} & \begin{minipage}[b]{\linewidth}\raggedright
Science
\end{minipage} \\
\begin{minipage}[b]{\linewidth}\raggedright
Evolution Biology and Society
\end{minipage} & \begin{minipage}[b]{\linewidth}\raggedright
Bio
\end{minipage} & \begin{minipage}[b]{\linewidth}\raggedright
Sex and Gender
\end{minipage} & \begin{minipage}[b]{\linewidth}\raggedright
Gender
\end{minipage} \\
\begin{minipage}[b]{\linewidth}\raggedright
Family
\end{minipage} & \begin{minipage}[b]{\linewidth}\raggedright
Family
\end{minipage} & \begin{minipage}[b]{\linewidth}\raggedright
Sexualities
\end{minipage} & \begin{minipage}[b]{\linewidth}\raggedright
Sex
\end{minipage} \\
\begin{minipage}[b]{\linewidth}\raggedright
Global and Transnational Sociology
\end{minipage} & \begin{minipage}[b]{\linewidth}\raggedright
Global
\end{minipage} & \begin{minipage}[b]{\linewidth}\raggedright
Social Psychology
\end{minipage} & \begin{minipage}[b]{\linewidth}\raggedright
SocPsych
\end{minipage} \\
\begin{minipage}[b]{\linewidth}\raggedright
History of Sociology and Social Thought
\end{minipage} & \begin{minipage}[b]{\linewidth}\raggedright
HistSoc
\end{minipage} & \begin{minipage}[b]{\linewidth}\raggedright
Sociological Practice and Public Sociology
\end{minipage} & \begin{minipage}[b]{\linewidth}\raggedright
Practice
\end{minipage} \\
\begin{minipage}[b]{\linewidth}\raggedright
Human Rights
\end{minipage} & \begin{minipage}[b]{\linewidth}\raggedright
HumRights
\end{minipage} & \begin{minipage}[b]{\linewidth}\raggedright
Teaching and Learning in Sociology
\end{minipage} & \begin{minipage}[b]{\linewidth}\raggedright
Teach
\end{minipage} \\
\begin{minipage}[b]{\linewidth}\raggedright
Indigenous Peoples and Native Nations
\end{minipage} & \begin{minipage}[b]{\linewidth}\raggedright
Indig
\end{minipage} & \begin{minipage}[b]{\linewidth}\raggedright
Theory
\end{minipage} & \begin{minipage}[b]{\linewidth}\raggedright
Theory
\end{minipage} \\
\begin{minipage}[b]{\linewidth}\raggedright
Inequality Poverty and Mobility
\end{minipage} & \begin{minipage}[b]{\linewidth}\raggedright
Ineq
\end{minipage} & \begin{minipage}[b]{\linewidth}\raggedright
\end{minipage} & \begin{minipage}[b]{\linewidth}\raggedright
\end{minipage} \\
\midrule()
\endhead
\bottomrule()
\end{longtable}

\bibliographystyle{unsrtnat}
\bibliography{references}  %%% Uncomment this line and comment out the ``thebibliography'' section below to use the external .bib file (using bibtex) .


%%% Uncomment this section and comment out the \bibliography{references} line above to use inline references.
% \begin{thebibliography}{1}

% 	\bibitem{kour2014real}
% 	George Kour and Raid Saabne.
% 	\newblock Real-time segmentation of on-line handwritten arabic script.
% 	\newblock In {\em Frontiers in Handwriting Recognition (ICFHR), 2014 14th
% 			International Conference on}, pages 417--422. IEEE, 2014.

% 	\bibitem{kour2014fast}
% 	George Kour and Raid Saabne.
% 	\newblock Fast classification of handwritten on-line arabic characters.
% 	\newblock In {\em Soft Computing and Pattern Recognition (SoCPaR), 2014 6th
% 			International Conference of}, pages 312--318. IEEE, 2014.

% 	\bibitem{hadash2018estimate}
% 	Guy Hadash, Einat Kermany, Boaz Carmeli, Ofer Lavi, George Kour, and Alon
% 	Jacovi.
% 	\newblock Estimate and replace: A novel approach to integrating deep neural
% 	networks with existing applications.
% 	\newblock {\em arXiv preprint arXiv:1804.09028}, 2018.

% \end{thebibliography}


\end{document}
